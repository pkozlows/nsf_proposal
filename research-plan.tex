\begin{enumerate}[label=\textbf{Aim \arabic*}]
    \item The uniform electron gas (UEG) is a paradigmatic system in condensed matter physics. As part of my rotation project with Prof. Lee, I am implementing fully self-consistent $GW$ (scGW) for the UEG. Now, I do not have prior experience with the UEG or scGW. However, in past research, I ran quantum chemistry calculations on a periodic system, like the UEG, and completed a senior thesis project on the $G_0W_0$ method within the same $GW$ approximation that scGW follows. I will first corroborate the reported result \cite{holm_fully_1998}, where scGW misses a satellite peak in the frequency spectrum found by highly accurate Quantum Monte Carlo (QMC) simulations.
    \item In the literature, expensive vertex corrections going beyond the $GW$ approximation have been shown to reproduce this satellite peak. I will see if MZ offers a cheaper solution, comparing the number of terms in its perturbative expansion needed to achieve an accuracy on par with vertex-corrected $GW$ for the UEG. I am prepared to work with the MZ theory of open quantum systems from experience with a recent class on quantum many body physics, where I learned to apply matrix product state ideas, culminating in my implementation of the Time Block Evolution Decimation (TEBD) algorithm. More generally, this exploration of MZ on the UEG will enable me to draw connections between MZ and $GW$ for solids, building upon previous work connecting diagrams between $GW$ and wave function-based methods for molecules \cite{Lange_2018}.

    \item I will apply the MZ framework to a realistic condensed matter system. The UEG is known to provide a good description of the Fermi sea in metallic systems, so the transition should be natural. The eventual goal is to study strongly correlated semiconductors composing photovoltaic systems with the MZ framework.
\end{enumerate}