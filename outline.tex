Solar energy has the momentum to replace fossil fuels in the green transition. To continue its propagation, the discovery of more efficient photovoltaic materials is necessary. Core spectroscopy has long been used to elucidate the electronic structure of materials. However, recently, it has been proven that this can be done on the attosecond time scale, for which the Nobel Prize was awarded in 2023. Such a short time scale lends high resolution to molecular processes, which allows for design of improved solar materials. Spectroscopists using this technique inform their experiments with computation. For this computation to be useful, quantum chemists need to improve the status quo, which poorly treats the strongly correlated electrons in photovoltaics.
\begin{wrapfigure}{r}{0.5\textwidth}
   \centering
   \includegraphics[width=0.48\textwidth]{picture.png}
   \caption{Impact of computational development.}
   \label{fig:fig1}
\end{wrapfigure}
Therefore, figure \ref{fig:fig1} shows how (1) the development of computational methodologies can (2) aid in the spectroscopy of materials to (3) produce more efficient solar cells. Density Functional Theory (DFT) has long served as the go-to computational method in materials science, due to its reasonable accuracy at low computational expense. However, it treats the repulsive interactions between electrons using an approximate exchange-correlation functional, leading to variable results with a lack of systematic improvability \cite{kozlowski_elucidating_2021}. A potential solution is the application of Green's functions in many-body perturbation theory (MBPT). Central is the Dyson equation,
\begin{equation}
G = G_0 + G_0 \Sigma G,
\label{eqn:dyson}
\end{equation}
which relates the Green's function of the fully interacting system \( G \) to that of the non-interacting system \( G_0 \) through the self-energy \( \Sigma \), which is designed to capture the many-body interactions neglected by \( G_0 \). Hedin provided a closed set of 5 equations that can be used to obtain \( G \) and \( \Sigma \). In the common \( GW \) approximation of Hedin's framework, the self-energy \( \Sigma \) takes the form \( iGW \), where \( W \) is the screened Coulomb interaction. Various levels of self-consistency can be done within \( GW \). The most basic is the one-shot \( G_0W_0 \), where the self-energy is calculated using the non-interacting Green's function and the bare Coulomb potential, i.e., \( \Sigma = iG_0W_0 \). Surprisingly, \( G_0W_0 \) has been shown to be quite accurate even at a modest computational cost, due to a fortuitous cancellation of errors. However, it shows a strong starting point dependence on \( G_0 \), which again implies the lack of systematic improvability. At the other extreme, we have fully self-consistent \( GW \) (scGW). Even at the high computational expense necessitated by full self-consistency, this scheme often does not deliver improved results over \( G_0W_0 \). To remedy this, one has to include vertex corrections beyond the \( GW \) approximation within Hedin's framework \cite{kutepov_one-electron_2017}, which is computationally intensive. The root of these issues is that Hedin's equations solve for the self-energy \( \Sigma \) through a perturbative expansion in the interaction strength \( W \). The \( GW \) approximation is accurate for weakly correlated systems where this expansion is reasonable, but it is not for the strongly correlated, where the interaction is large.

The Mori-Zwanzig (MZ) theory offers an alternative. Originating from statistical physics, one starts from a similar Dyson equation \ref{eqn:dyson}, but the self-energy \( \Sigma \) is replaced by a memory kernel. This memory kernel is expanded in powers of the evolution time \( t \), whereas in \( GW \) the self-energy is expanded in powers of the interaction strength \( W \). This makes the series expansion of Dyson's equation converge faster with MZ for strongly correlated systems. Recently, a diagrammatic theory for MZ has been introduced in the form of tree diagrams \cite{zhu_combinatorial_2022}, as opposed to the Feynman diagrams of \( GW \), suggesting the potential for a computational implementation. However, this has not been done yet and is the goal of the proposed work. 